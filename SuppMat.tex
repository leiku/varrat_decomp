\documentclass[letterpaper,11pt]{article}

\usepackage{amsmath}
\usepackage{amsfonts}
\usepackage{graphicx}
\usepackage{natbib}
\usepackage[top=1.25cm, bottom=1.25cm, left=1.25cm, right=1.25cm]{geometry}
\usepackage{amsthm}
\usepackage{lscape}

\newcommand{\var}{{\operatorname{var}}}
\newcommand{\cov}{{\operatorname{cov}}}
\newcommand{\cor}{{\operatorname{cor}}}
\newcommand{\E}{{\operatorname{E}}}
\newcommand{\mean}{{\operatorname{mean}}}

\newcommand{\CV}{{\operatorname{CV}}}
\newcommand{\com}{{\operatorname{com}}}
\newcommand{\comip}{{\operatorname{com\_ip}}}
\newcommand{\td}{{\operatorname{td}}}
\newcommand{\ts}{{\operatorname{ts}}}

\renewcommand{\qedsymbol}{\rule{.7em}{.7em}}
\renewcommand{\thesection}{S\arabic{section}}
\renewcommand{\thetable}{S\arabic{table}}
\renewcommand{\thefigure}{S\arabic{figure}}

\newtheorem{theorem}{Theorem}[]
\newtheorem{corollary}[theorem]{Corollary}
\newtheorem{lemma}[theorem]{Lemma}
\newtheorem{proposition}[theorem]{Proposition}
\newtheorem{definition}[theorem]{Definition}

\begin{document}

\title{Appendices for: A new variance ratio metric to detect the timescale of compensatory dynamics}
\author{Zhao, Wang, Hallett, Rypel, Castorani, Shoemaker, Cottingham, Suding, Reuman}
\date{}
\maketitle

\section{Details of spectral methods}\label{sec:spec}
\noindent Dan to fill this in

\section{Details of the theory}
\noindent We recapitulate definitions from the main text. Let $x_i(t)$ be population 
density or biomass time series for times $t=1,\ldots,T$ and taxa $i=1,\ldots,S$, and 
let $\mu_i=\mean(x_i (t))$, $v_{ij}=\cov(x_i (t),y_j (t))$, $x_{\text{tot}} (t)=\sum_i x_i(t)$, $\mu_{\text{tot}}=\mean(x_{\text{tot}}(t))=\sum_i \mu_i$, and $v_{\text{tot}}=\var(x_{\text{tot}}(t))=\sum_{i,j}v_{ij}$. Then, using the definitions of spectral quantities provided in Appendix \ref{sec:spec}, we define
\begin{align}
\text{CV}_{\text{com}}^2 &= \frac{v_{\text{tot}}}{\mu_{\text{tot}}^2}=\frac{\sum_{i,j}v_{ij}}{(\sum_i \mu_i)^2},\\
\text{CV}_{\text{com\_ip}}^2 &=\frac{\sum_{i}v_{ii}}{(\sum_i \mu_i)^2}
=\frac{\sum_{i}v_{ii}}{\mu_{\text{tot}}^2},\\
\varphi &= \frac{v_{\text{tot}}}{\sum_i v_{ii}}
=\frac{\sum_{i,j}v_{ij}}{\sum_i v_{ii}}, \\
\text{CV}_{\text{com}}^2(\sigma) &= \frac{s_{\text{tt}}(\sigma)}{\mu_{\text{tot}}^2}
=\frac{\sum_{i,j} s_{ij}(\sigma)}{(\sum_i \mu_{i})^2},\\
\text{CV}_{\text{com\_ip}}^2(\sigma) &= \frac{\sum_i s_{ii}(\sigma)}{(\sum_i \mu_{i})^2}
=\frac{\sum_{i}s_{ii}(\sigma)}{\mu_{\text{tot}}^2},\\
\varphi_{\text{ts}}(\sigma) &= \frac{s_{\text{tt}}(\sigma)}{\sum_i s_{ii}(\sigma)}
=\frac{\sum_{i,j}s_{ij}(\sigma)}{\sum_i s_{ii}(\sigma)},\\
w(\sigma) &= \frac{\sum_i s_{ii}(\sigma)}{\sum_i v_{ii}},\\
\overline{\varphi_{\text{ts}}}(\Omega) &= \frac{\sum_{\sigma \in \Omega} \varphi_{\text{ts}}(\sigma) w(\sigma)}
{\sum_{\sigma \in \Omega} w(\sigma)}.
\end{align}
The quantities $\mu_{\text{tot}}^2$, $\sum_i v_{ii}$ and 
$\sum_i s_{ii}(\sigma)$ appearing in denominators above 
should typically be nonzero for all real data sets. 

The above definitions lead to two theorems as follows:
\begin{theorem}\label{thm:freqdecom}
Assuming $\mu_{\text{tot}}$ and $\sum_i v_{ii}$ are nonzero, the following timescale decompositions hold exactly, where summations are over the
timescales $\sigma=\frac{T}{T-1},\frac{T}{T-2},\ldots,\frac{T}{2},T$:
\begin{align}
\sum_{\sigma}\CV_{\com}^2(\sigma) &= \CV_{\com}^2, \label{eq:th1eq1}\\
\sum_{\sigma}\CV_{\comip}^2(\sigma) &= \CV_{\comip}^2, \label{eq:th1eq2}\\
\sum_{\sigma}\varphi_{\ts}(\sigma) w(\sigma) &= \varphi. \label{eq:th1eq3}
\end{align}
\end{theorem}
\begin{proof}
Equation (\ref{eq:th1eq1}) follows because
\begin{align}
\sum_{\sigma}\CV_{\com}^2(\sigma) &= \sum_{\sigma} \frac{s_{\text{tt}}(\sigma)}{\mu_{\text{tot}}^2} \\
&= \frac{\sum_{\sigma}  s_{\text{tt}}(\sigma)}{\mu_{\text{tot}}^2} \\
&= \frac{v_{\text{tot}}}{\mu_{\text{tot}}^2} \label{eq:pf1eq1}\\
&= \CV_{\com}^2,
\end{align}
where (\ref{eq:pf1eq1}) follows by the properties of the power spectrum (Appendix
\ref{sec:spec}). Equation (\ref{eq:th1eq2}) follows because
\begin{align}
\sum_{\sigma}\CV_{\comip}^2(\sigma) &= \sum_{\sigma} \frac{\sum_i s_{ii}(\sigma)}{\mu_{\text{tot}}^2} \\
&= \frac{\sum_i \sum_\sigma s_{ii}(\sigma)}{\mu_{\text{tot}}^2} \\
&= \frac{\sum_i v_{ii}}{\mu_{\text{tot}}^2} \label{eq:pf1eq2}\\
&= \CV_{\comip}^2,
\end{align}
where (\ref{eq:pf1eq2}) again follows by the properties of the power spectrum
(Appendix \ref{sec:spec}). Equation (\ref{eq:th1eq3}) follows because
\begin{align}
\sum_{\sigma}\varphi_{\ts}(\sigma) w(\sigma) &= \sum_{\sigma} \frac{s_{\text{tt}}(\sigma)}{\sum_i s_{ii}(\sigma)}
 \frac{\sum_i s_{ii}(\sigma)}{\sum_i v_{ii}} \\
&= \frac{\sum_{\sigma} s_{\text{tt}}(\sigma)}{\sum_i v_{ii}} \\
&= \frac{v_{\text{tot}}}{\sum_i v_{ii}} \label{eq:pf1eq3}\\
&= \varphi, 
\end{align}
where (\ref{eq:pf1eq3}) again follows by the properties of the power spectrum
(Appendix \ref{sec:spec}). 
\end{proof}

\begin{theorem}\label{thm:relate}
Assuming $\mu_{\text{tot}}$ and $\sum_i v_{ii}$ are nonzero,
\begin{equation}
\CV_{\com}^2 = \varphi \CV_{\comip}^2. \label{eq:th2eq1}\\
\end{equation}
Again assuming $\mu_{\text{tot}}$ and $\sum_i v_{ii}$ are nonzero, and for $\sigma$ such
that $\sum_i s_{ii}(\sigma) \neq 0$,
\begin{align}
\CV_{\com}^2(\sigma) &= \varphi_{\ts}(\sigma) \CV_{\comip}^2(\sigma). \label{eq:th2eq2}
\end{align}
\end{theorem}
\begin{proof}
Equation (\ref{eq:th2eq1}) holds because 
\begin{align}
\varphi \CV_{\comip}^2 &= \frac{v_{\text{tot}}}{\sum_i v_{ii}}
\frac{\sum_i v_{ii}}{\mu_{\text{tot}}^2} \\
&= \frac{v_{\text{tot}}}{\mu_{\text{tot}}^2} \\
&= \CV_{\com}^2. 
\end{align}
Equation (\ref{eq:th2eq2}) holds because, for $\sigma$ for which 
$\sum_i s_{ii}(\sigma) \neq 0$, we have
\begin{align}
\varphi_{\ts}(\sigma) \CV_{\comip}^2(\sigma) &= 
\frac{s_{\text{tt}}(\sigma)}{\sum_i s_{ii}(\sigma)}
\frac{\sum_i s_{ii}(\sigma)}{\mu_{\text{tot}}^2} \\
&= \frac{s_{\text{tt}}(\sigma)}{\mu_{\text{tot}}^2} \\
&= \CV_{\com}^2(\sigma).
\end{align}
\end{proof}

\begin{theorem}\label{thm:range}
Assuming $\mu_{\text{tot}}$ and $\sum_i v_{ii}$ are nonzero, and for $\sigma$ such
that $\sum_i s_{ii}(\sigma) \neq 0$,
\begin{align}
\CV_{\com}^2(\Omega) &= \CV_{\comip}^2(\Omega)  \overline{\varphi_{\text{ts}}}(\Omega) . \label{eq:the3eq1}
\end{align}
\end{theorem}
\begin{proof}
Equation (\ref{eq:the3eq1}) holds because
\begin{align}
\CV_{\comip}^2(\Omega)  \overline{\varphi_{\text{ts}}}(\Omega) 
&= \sum_{\sigma \in \Omega} \frac{\sum_i s_{ii}(\sigma)}{\mu_{\text{tot}}^2}  \frac{\sum_{\sigma \in \Omega} \varphi_{\text{ts}}(\sigma) w(\sigma)}{\sum_{\sigma \in \Omega} w_{\sigma}}  \\
&= \frac{\sum_{\sigma \in \Omega} \sum_i s_{ii}(\sigma)}{\mu_{\text{tot}}^2} \frac{\sum_{\sigma \in \Omega} \frac{s_{\text{tt}}(\sigma)}{\sum_i s_{ii}(\sigma)} \frac{\sum_i s_{ii}(\sigma)}{\sum_i v_{ii}}}{\sum_{\sigma \in \Omega} \frac{\sum_i s_{ii}(\sigma)}{\sum_i v_{ii}}} \\
&= \frac{\sum_{\sigma \in \Omega} \sum_i s_{ii}(\sigma)}{\mu_{\text{tot}}^2} \frac{\sum_{\sigma \in \Omega} s_{\text{tt}}(\sigma)}{\sum_i v_{ii}} \frac{\sum_i v_{ii}}{\sum_{\sigma \in \Omega} \sum_i s_{ii}(\sigma)} \\
&= \frac{\sum_{\sigma \in \Omega} s_{tt}(\sigma)}{\mu_{\text{tot}}^2} \\
&= \CV_{\com}^2(\Omega)
\end{align}
\end{proof}

\section{Connections to the variance ratio of Loreau and de Mazancourt}
\noindent Note to coauthors: this section currently holds some ideas.
My feeling is they may not be complete (Shaopeng, maybe you have some insights
to add?), and they are certainly not presented optimally. This section is
for storing the ideas right now.

The variance ratio of Loreau and de Mazancourt is 
\begin{equation}
\varphi^{(m)}=\frac{\sum_{i,j} v_{ij}}{(\sum_i \sqrt{v_{ii}})^2},
\end{equation}
and takes values between $0$ and $1$. The modified variance ratio 
can be related to the classic variance ratio via
\begin{align}
\varphi &= \frac{\sum_{i,j} v_{ij}}{\sum_i v_{ii}} \\
&= \frac{\sum_{i,j} v_{ij}}{(\sum_i \sqrt{v_{ii}})^2}
\frac{(\sum_i \sqrt{v_{ii}})^2}{\sum_i v_{ii}} \\
&= \varphi^{(m)} \frac{(\sum_i \sqrt{v_{ii}})^2}{\sum_i v_{ii}}.
\end{align}
The quantity 
\begin{equation}
f=\frac{(\sum_i \sqrt{v_{ii}})^2}{\sum_i v_{ii}}
\end{equation}
takes values between $1$ and $S$, and characterizes the degree to which the 
variances $v_{ii}$ are heterogeneous. For instance, when these variance are all
the same this term is $S$. When there is an index $i$ such that 
$v_{ii} \gg v_{jj}$ for all $j \neq i$, $\varphi^{(m)}$ is close to $1$. 
We note in particular that $f$ does not really relate to synchrony,
as it does not depend on covariances or correlations between time series in
different locations. This makes it clear that, while $\varphi$ has information
about synchrony in it, it has other information as well.

The previous relationship $\CV_{\com}^2=\varphi \CV_{\comip}^2$ becomes 
\begin{equation}
\CV_{\com}^2=\varphi^{(m)} f \CV_{\comip}^2.
\end{equation}
The quantity $\varphi^{(m)}$ is considered an index of synchrony, and this 
new equation makes it clear that not only is synchrony implicated in
whether population instability translates into community instability,
so is the quantity $f$, which represents the degree to which the variances
$v_{ii}$ are heterogeneous. But this may make sense because when these variances
are very different the degree to which time series can  reinforce each other
or cancel each other out is limited, whereas when the variances are very similar,
time series can reinforce each other a lot or cancel each other out a lot,
so $f$ is large and the effects of the overall variance ratio
$\varphi$ are accentuated. I feel as though more can be said here, or it 
can be said better. I am nervous about the fact that $\varphi^{(m)}$
is not purely about synchrony either - I feel like if it were purely 
about synchrony, it should only depend on the correlations between
time series. Maybe there is yet another version of the variance ratio. 
I have tried to develop one that only depends on correlations between the time
series, with little success.

The timescale-specific variance ratio also decomposes in a similar way:
\begin{equation}
\varphi_{\text{ts}}(\sigma)=\frac{\sum_{i,j}s_{ij}(\sigma)}{(\sum_i \sqrt{s_{ii}(\sigma)})^2}
\frac{(\sum_i \sqrt{s_{ii}(\sigma)})^2}{\sum_i s_{ii}(\sigma)}.
\end{equation}
The first factor on the right here is a timescale-specific version of 
$\varphi^{(m)}$. The new term
\begin{equation}
f(\sigma)=\frac{(\sum_i \sqrt{s_{ii}(\sigma)})^2}{\sum_i s_{ii}(\sigma)}
\end{equation}
is a timescale-specific version of $f$, with similar interpretation.
And it should have the same bounds, between $1$ and $S$.

\newpage
\clearpage

%\begin{landscape}
% latex table generated in R 3.5.1 by xtable 1.8-3 package
% Mon Sep 23 10:20:13 2019
\begin{table}[ht]
\centering
\scalebox{0.75}{
\begin{tabular}{lllllllll}
  \hline
Site & Abbr. & Years & Yr. rg. & Plots & Plot size & Richness & Measured & Description \\ 
  \hline
Jasper Ridge Biological Preserve & JRG &  28 & 1983-2010 &  18 & 0.80 & 25.70 & Percent cover & Serpentine grassland \\ 
  Kellogg Biological Station LTER & KBS &  11 & 1999-2009 &  30 & 1.00 & 34.50 & Biomass & Old field \\ 
  Hays, Kansas & HAY &  30 & 1943-1972 &  13 & 1.00 & 22.20 & Percent cover & Tallgrass prairie \\ 
  Jornada Basin LTER & JRN &  20 & 1989-2008 &  47 & 1.00 & 28.50 & Allometric biomass & Desert grassland \\ 
  Konza Prarie LTER & KNZ &  24 & 1983-2006 &  20 & 10.00 & 37.60 & Percent cover & Annually burned tallgrass prairie \\ 
  Sevilleta LTER & SEV &  13 & 1999-2011 &  22 & 1.00 & 13.40 & Biomass & Desert grassland \\ 
   \hline
\end{tabular}
}
\caption[Summary of datasets]{Summary of datasets. Plot size in square meters. Richness is the number of species that were ever seen in a plot, averaged across plots for a site. Biomass, when measured, was in g per square meter} 
\end{table}

%\end{landscape}

% Figure
\begin{figure}
\includegraphics[width=0.9\textwidth]{Figs/fig_decomposed_1_classic_jrg_sorted_vrf.pdf}
\caption{Demonstration of timescale analyses of data from JRG, which consist of 18 plots (indicated by different colors in left panels, and by x-axis in right panels). Different colors in the right panels indicate the average (or weighted average) of the values in the corresponding left panels across short timescales ($<$ 4 years; blue) or long timescales ($\geq$ 4 years; red). On right panels, plots are sorted on the x-axis by the difference between short- and long-timescale averaged $\varphi _{ts} (\sigma)$.}
\end{figure}

\begin{figure}
\includegraphics[width=0.9\textwidth]{Figs/fig_decomposed_2_classic_kbs_sorted_vrf.pdf}
\caption{Demonstration of timescale analyses of data from KBS, which consist of 30 plots (indicated by different colors in left panels, and by x-axis in right panels). Different colors in the right panels indicate the average (or weighted average) of the values in the corresponding left panels across short timescales ($<$ 4 years; blue) or long timescales ($\geq$ 4 years; red). On right panels, plots are sorted on the x-axis by the difference between short- and long-timescale averaged $\varphi _{ts} (\sigma)$.}
\end{figure}

\begin{figure}
\includegraphics[width=0.9\textwidth]{Figs/fig_decomposed_3_classic_hay_sorted_vrf.pdf}
\caption{Demonstration of timescale analyses of data from HAY, which consist of 13 plots (indicated by different colors in left panels, and by x-axis in right panels). Different colors in the right panels indicate the average (or weighted average) of the values in the corresponding left panels across short timescales ($<$ 4 years; blue) or long timescales ($\geq$ 4 years; red). On right panels, plots are sorted on the x-axis by the difference between short- and long-timescale averaged $\varphi _{ts} (\sigma)$.}
\end{figure}

\begin{figure}
\includegraphics[width=0.9\textwidth]{Figs/fig_decomposed_4_classic_jrn_omit_sorted_vrf.pdf}
\caption{Demonstration of timescale analyses of data from JRN, which consist of 47 plots (indicated by different colors in left panels, and by x-axis in right panels). Different colors in the right panels indicate the average (or weighted average) of the values in the corresponding left panels across short timescales ($<$ 4 years; blue) or long timescales ($\geq$ 4 years; red). On right panels, plots are sorted on the x-axis by the difference between short- and long-timescale averaged $\varphi _{ts} (\sigma)$.}
\end{figure}

\begin{figure}
\includegraphics[width=0.9\textwidth]{Figs/fig_decomposed_5_classic_knz_sorted_vrf.pdf}
\caption{Demonstration of timescale analyses of data from KNZ, which consist of 20 plots (indicated by different colors in left panels, and by x-axis in right panels). Different colors in the right panels indicate the average (or weighted average) of the values in the corresponding left panels across short timescales ($<$ 4 years; blue) or long timescales ($\geq$ 4 years; red). On right panels, plots are sorted on the x-axis by the difference between short- and long-timescale averaged $\varphi _{ts} (\sigma)$.}
\end{figure}

\begin{figure}
\includegraphics[width=0.9\textwidth]{Figs/fig_decomposed_6_classic_sev_sorted_vrf.pdf}
\caption{Demonstration of timescale analyses of data from SEV, which consist of 22 plots (indicated by different colors in left panels, and by x-axis in right panels). Different colors in the right panels indicate the average (or weighted average) of the values in the corresponding left panels across short timescales ($<$ 4 years; blue) or long timescales ($\geq$ 4 years; red). On right panels, plots are sorted on the x-axis by the difference between short- and long-timescale averaged $\varphi _{ts} (\sigma)$.}
\end{figure}

\begin{figure}
\includegraphics[width=0.9\textwidth]{Figs/fig_timescale_quantile.pdf}
\caption{Distributions across plots within sites of timescale-specific variance ratios
$\varphi_{\text{ts}}(\sigma)$. The black line is the median of the distribution
across plots of $\varphi_{\text{ts}}(\sigma)$ for each timescale, $\sigma$. Dark-gray
shading covers $25^{\text{th}}$ and $75^{\text{th}}$ quantiles, and light-gray 
shading covers $2.5^{\text{th}}$ and $97.5^{\text{th}}$ quantiles. The horizontal 
dashed line is a variance ratio value of $1$.}
\end{figure}

\begin{figure}
\includegraphics[width=0.9\textwidth]{Figs/fig_new_vs_classic.pdf}
\caption{Relationship between the weighted average of $\varphi_{\text{ts}}(\sigma)$
across short (blue) and long (red) timescales and the classic, non-timescale-specific
variance ratio across plots for each site. The dashed line on each panel is the 1:1 
line.}
\end{figure}

\end{document}


