\documentclass[letterpaper,11pt]{article}

\usepackage{amsmath}
\usepackage{amsfonts}
\usepackage{graphicx}
\usepackage{natbib}
\usepackage[top=1.25cm, bottom=1.25cm, left=1.25cm, right=1.25cm]{geometry}
\usepackage{amsthm}
\usepackage{lscape}

\newcommand{\var}{{\operatorname{var}}}
\newcommand{\cov}{{\operatorname{cov}}}
\newcommand{\cor}{{\operatorname{cor}}}
\newcommand{\E}{{\operatorname{E}}}
\newcommand{\mean}{{\operatorname{mean}}}

\newcommand{\CV}{{\operatorname{CV}}}
\newcommand{\com}{{\operatorname{com}}}
\newcommand{\comip}{{\operatorname{com\_ip}}}
\newcommand{\pop}{{\operatorname{pop}}}
\newcommand{\td}{{\operatorname{td}}}
\newcommand{\ts}{{\operatorname{ts}}}
\newcommand{\Rp}{{\operatorname{Re}}}

\renewcommand{\qedsymbol}{\rule{.7em}{.7em}}
\renewcommand{\thesection}{S\arabic{section}}
\renewcommand{\thetable}{S\arabic{table}}
\renewcommand{\thefigure}{S\arabic{figure}}

\newtheorem{theorem}{Theorem}[]
\newtheorem{corollary}[theorem]{Corollary}
\newtheorem{lemma}[theorem]{Lemma}
\newtheorem{proposition}[theorem]{Proposition}
\newtheorem{definition}[theorem]{Definition}

\begin{document}

\title{Appendices for: A new variance ratio metric to detect the timescale of compensatory dynamics}
\author{Zhao, Wang, Hallett, Rypel, Sheppard, Castorani, Shoemaker, Cottingham, Suding, Reuman}
\date{}
\maketitle

\section{Basics of Fourier methods}\label{sec:spec}
\noindent If $x_i(t)$ is a population abundance measurement for taxon $i$ 
($i=1,\ldots,S$) at time $t$ ($t=1,\ldots,T$), the discrete Fourier transform is 
\begin{equation}
X_i(h)=\sum_{t=1}^T x_i(t) \exp(-2 \pi i (t-1)(h-1)/T),
\end{equation}
for $h=1,\ldots,T$. The value $X_i(1)$ equals the sum of the time series $x_i(t)$,
and we discard that value. The value $X_i(h)$ for $h>1$ is the Fourier transform 
at frequency $(h-1)/T$ (units of cycles per sampling interval), or timescale 
$\sigma=T/(h-1)$ (units of sampling intervals), which we henceforth
denote $X_i(\sigma)$ for $\sigma=T/(T-1),T/(T-2),\ldots,T/2,T$.

We define $s_{ii}(\sigma)=\bar{X}_i(\sigma) X_i(\sigma)/(T(T-1))$, where the overbar
represents complex conjugation. If $x_i(t)$
can be assumed to be a finite sample from a stationary stochastic 
process, then this is the periodogram estimate of the power spectrum of that
process \citep{Brillinger2001}. We define 
$s_{ij}(\sigma)=\Rp(\bar{X}_i(\sigma) X_j(\sigma))/(T(T-1))$. If the $x_i(t)$
and $x_j(t)$ together can be assumed to be a finite sample from a two-dimensional
stationary stochastic process, then this is the periodogram estimate of the 
cospectrum of that process \citep{Brillinger2001}. 

We now state some facts of basic Fourier analysis without proof. There are many 
references, including the classic book of \cite{Brillinger2001}, that provide proofs
and extensive further background on the statistics of spectral analysis of time series.
First, $s_{ii}(\sigma)$ and $s_{ij}(\sigma)$ are real-valued quantities, and 
$\sum_\sigma s_{ii}(\sigma) = \var(x_i(t))$ and 
$\sum_\sigma s_{ij}(\sigma) = \cov(x_i(t),x_j(t))$, where the summations are over
$\sigma=T/(T-1),T/(T-2),\ldots,T/2,T$. Second, $s_{ii}(\sigma)$ and $s_{ij}(\sigma)$
are symmetric about the ``Nyquist timescale'' (2 time steps), i.e., 
the vector of those elements of $s_{ii}(\sigma)$ (respectively, $s_{ij}(\sigma)$)
for which $\sigma<2$ equals the vector of those elements of 
$s_{ii}(\sigma)$ (respectively, $s_{ij}(\sigma)$) for which $\sigma>2$, written
in reverse order. This is why plots of these quantities against timescale are 
symmetric. Typically one plots and interprets 
these quantities only for timescales $\sigma \geq 2$,
but we plot all timescales in Fig. 3 in the main text 
because sums and averages over all timescales are 
required to recover non-timescale-specific variance ratios and population
and community variability statistics (see next section), and part of the 
purpose of that figure is to illustrate that these sums and averages do
recover the non-timescale-specific quantities.

\section{Details of the theory}
\noindent We recapitulate definitions from the main text. If $x_i(t)$ is a population abundance measurement for taxon $i$ 
($i=1,\ldots,S$) at time $t$ ($t=1,\ldots,T$), then let 
$\mu_i=\mean(x_i (t))$, $v_{ij}=\cov(x_i (t),y_j (t))$, $x_{\text{tot}} (t)=\sum_i x_i(t)$, 
$\mu_{\text{tot}}=\mean(x_{\text{tot}}(t))=\sum_i \mu_i$, and 
$v_{\text{tot}}=\var(x_{\text{tot}}(t))=\sum_{i,j}v_{ij}$. Then, 
using the definitions of spectral quantities provided in Appendix \ref{sec:spec} and 
defining $s_{\text{tt}}(\sigma)$ to be the periodogram estimate of the 
power spectrum of $x_{\text{tot}}(t)$ and $\Omega$ to be a set of timescales, we define
\begin{align}
\text{CV}_{\text{com}}^2 &= \frac{v_{\text{tot}}}{\mu_{\text{tot}}^2}=\frac{\sum_{i,j}v_{ij}}{(\sum_i \mu_i)^2},\\
\text{CV}_{\text{com\_ip}}^2 &=\frac{\sum_{i}v_{ii}}{(\sum_i \mu_i)^2}
=\frac{\sum_{i}v_{ii}}{\mu_{\text{tot}}^2},\\
\varphi &= \frac{v_{\text{tot}}}{\sum_i v_{ii}}
=\frac{\sum_{i,j}v_{ij}}{\sum_i v_{ii}}, \\
\text{CV}_{\text{com}}^2(\sigma) &= \frac{s_{\text{tt}}(\sigma)}{\mu_{\text{tot}}^2}
=\frac{\sum_{i,j} s_{ij}(\sigma)}{(\sum_i \mu_{i})^2},\\
\text{CV}_{\text{com\_ip}}^2(\sigma) &= \frac{\sum_i s_{ii}(\sigma)}{(\sum_i \mu_{i})^2}
=\frac{\sum_{i}s_{ii}(\sigma)}{\mu_{\text{tot}}^2},\\
\varphi_{\text{ts}}(\sigma) &= \frac{s_{\text{tt}}(\sigma)}{\sum_i s_{ii}(\sigma)}
=\frac{\sum_{i,j}s_{ij}(\sigma)}{\sum_i s_{ii}(\sigma)},\\
w(\sigma) &= \frac{\sum_i s_{ii}(\sigma)}{\sum_i v_{ii}},\\
\text{CV}_{\text{com}}^2(\Omega) &= \sum_{\sigma \in \Omega} \text{CV}_{\text{com}}^2(\sigma),\\
\text{CV}_{\text{com\_ip}}^2(\Omega) &= \sum_{\sigma \in \Omega} \text{CV}_{\text{com\_ip}}^2(\sigma),\\
\overline{\varphi_{\text{ts}}}(\Omega) &= \frac{\sum_{\sigma \in \Omega} \varphi_{\text{ts}}(\sigma) w(\sigma)}
{\sum_{\sigma \in \Omega} w(\sigma)}.
\end{align}
The quantities $\mu_{\text{tot}}^2$, $\sum_i v_{ii}$ and 
$\sum_i s_{ii}(\sigma)$ appearing in denominators above 
should typically be nonzero for all real data sets. 

The above definitions lead to three theorems as follows:
\begin{theorem}\label{thm:freqdecom}
Assuming $\mu_{\text{tot}}$ and $\sum_i v_{ii}$ are nonzero, the following timescale decompositions hold exactly, where summations are over the
timescales $\sigma=\frac{T}{T-1},\frac{T}{T-2},\ldots,\frac{T}{2},T$:
\begin{align}
\sum_{\sigma}\CV_{\com}^2(\sigma) &= \CV_{\com}^2, \label{eq:th1eq1}\\
\sum_{\sigma}\CV_{\comip}^2(\sigma) &= \CV_{\comip}^2, \label{eq:th1eq2}\\
\sum_{\sigma}\varphi_{\ts}(\sigma) w(\sigma) &= \varphi. \label{eq:th1eq3}
\end{align}
\end{theorem}
\begin{proof}
Equation (\ref{eq:th1eq1}) follows because
\begin{align}
\sum_{\sigma}\CV_{\com}^2(\sigma) &= \sum_{\sigma} \frac{s_{\text{tt}}(\sigma)}{\mu_{\text{tot}}^2} \\
&= \frac{\sum_{\sigma}  s_{\text{tt}}(\sigma)}{\mu_{\text{tot}}^2} \\
&= \frac{v_{\text{tot}}}{\mu_{\text{tot}}^2} \label{eq:pf1eq1}\\
&= \CV_{\com}^2,
\end{align}
where (\ref{eq:pf1eq1}) follows by the properties of the power spectrum 
estimate (Appendix
\ref{sec:spec}). Equation (\ref{eq:th1eq2}) follows because
\begin{align}
\sum_{\sigma}\CV_{\comip}^2(\sigma) &= \sum_{\sigma} \frac{\sum_i s_{ii}(\sigma)}{\mu_{\text{tot}}^2} \\
&= \frac{\sum_i \sum_\sigma s_{ii}(\sigma)}{\mu_{\text{tot}}^2} \\
&= \frac{\sum_i v_{ii}}{\mu_{\text{tot}}^2} \label{eq:pf1eq2}\\
&= \CV_{\comip}^2,
\end{align}
where (\ref{eq:pf1eq2}) again follows by the properties of the power spectrum
estimate we use (Appendix \ref{sec:spec}). Equation (\ref{eq:th1eq3}) follows because
\begin{align}
\sum_{\sigma}\varphi_{\ts}(\sigma) w(\sigma) &= \sum_{\sigma} \frac{s_{\text{tt}}(\sigma)}{\sum_i s_{ii}(\sigma)}
 \frac{\sum_i s_{ii}(\sigma)}{\sum_i v_{ii}} \\
&= \frac{\sum_{\sigma} s_{\text{tt}}(\sigma)}{\sum_i v_{ii}} \\
&= \frac{v_{\text{tot}}}{\sum_i v_{ii}} \label{eq:pf1eq3}\\
&= \varphi, 
\end{align}
where (\ref{eq:pf1eq3}) again follows by the properties of the power spectrum
estimate (Appendix \ref{sec:spec}). There is an assumption used here that 
$\sum_i s_{ii}(\sigma)$ differs from $0$ for all timescales 
$\sigma$, which should typically be true for real datasets. However, for
datasets for which $\sum_i s_{ii}(\sigma)$ is $0$ for some $\sigma$ (e.g., Fig. 1 in 
the main text),
the summation in equation \ref{eq:th1eq3} can be replaced by a summation
over timescales for which $\sum_i s_{ii}(\sigma) \neq 0$ to recover the
non-timescale-specific variance ratio (Fig. 3, main text).
\end{proof}

\begin{theorem}\label{thm:relate}
Assuming $\mu_{\text{tot}}$ and $\sum_i v_{ii}$ are nonzero,
\begin{equation}
\CV_{\com}^2 = \varphi \CV_{\comip}^2. \label{eq:th2eq1}\\
\end{equation}
Again assuming $\mu_{\text{tot}}$ and $\sum_i v_{ii}$ are nonzero, and for $\sigma$ such
that $\sum_i s_{ii}(\sigma) \neq 0$,
\begin{align}
\CV_{\com}^2(\sigma) &= \varphi_{\ts}(\sigma) \CV_{\comip}^2(\sigma). \label{eq:th2eq2}
\end{align}
\end{theorem}
\begin{proof}
Equation (\ref{eq:th2eq1}) holds because 
\begin{align}
\varphi \CV_{\comip}^2 &= \frac{v_{\text{tot}}}{\sum_i v_{ii}}
\frac{\sum_i v_{ii}}{\mu_{\text{tot}}^2} \\
&= \frac{v_{\text{tot}}}{\mu_{\text{tot}}^2} \\
&= \CV_{\com}^2. 
\end{align}
Equation (\ref{eq:th2eq2}) holds because, for $\sigma$ for which 
$\sum_i s_{ii}(\sigma) \neq 0$, we have
\begin{align}
\varphi_{\ts}(\sigma) \CV_{\comip}^2(\sigma) &= 
\frac{s_{\text{tt}}(\sigma)}{\sum_i s_{ii}(\sigma)}
\frac{\sum_i s_{ii}(\sigma)}{\mu_{\text{tot}}^2} \\
&= \frac{s_{\text{tt}}(\sigma)}{\mu_{\text{tot}}^2} \\
&= \CV_{\com}^2(\sigma).
\end{align}
\end{proof}

\begin{theorem}\label{thm:range}
Assuming $\mu_{\text{tot}}$ and $\sum_i v_{ii}$ are nonzero, and for $\Omega$ only 
containing $\sigma$ such that $\sum_i s_{ii}(\sigma) \neq 0$,
\begin{align}
\CV_{\com}^2(\Omega) &= \CV_{\comip}^2(\Omega)  \overline{\varphi_{\text{ts}}}(\Omega) . \label{eq:the3eq1}
\end{align}
\end{theorem}
\begin{proof}
\begin{align}
\CV_{\comip}^2(\Omega)  \overline{\varphi_{\text{ts}}}(\Omega) 
&= \left[ \sum_{\sigma \in \Omega} \frac{\sum_i s_{ii}(\sigma)}{\mu_{\text{tot}}^2} \right]  \frac{\sum_{\sigma \in \Omega} \varphi_{\text{ts}}(\sigma) w(\sigma)}{\sum_{\sigma \in \Omega} w_{\sigma}}  \\
&= \left[ \frac{\sum_{\sigma \in \Omega} \sum_i s_{ii}(\sigma)}{\mu_{\text{tot}}^2} \right] \frac{\sum_{\sigma \in \Omega} \frac{s_{\text{tt}}(\sigma)}{\sum_i s_{ii}(\sigma)} \frac{\sum_i s_{ii}(\sigma)}{\sum_i v_{ii}}}{\sum_{\sigma \in \Omega} \frac{\sum_i s_{ii}(\sigma)}{\sum_i v_{ii}}} \\
&= \frac{\sum_{\sigma \in \Omega} \sum_i s_{ii}(\sigma)}{\mu_{\text{tot}}^2} \frac{\sum_{\sigma \in \Omega} s_{\text{tt}}(\sigma)}{\sum_i v_{ii}} \frac{\sum_i v_{ii}}{\sum_{\sigma \in \Omega} \sum_i s_{ii}(\sigma)} \\
&= \frac{\sum_{\sigma \in \Omega} s_{\text{tt}}(\sigma)}{\mu_{\text{tot}}^2} \\
&= \CV_{\com}^2(\Omega)
\end{align}
\end{proof}

\section{Connections to the variance ratio of Loreau and de Mazancourt}
\noindent The variance ratio of Loreau and de Mazancourt is 
\begin{equation}
\varphi^{(m)}=\frac{\sum_{i,j} v_{ij}}{(\sum_i \sqrt{v_{ii}})^2},
\end{equation}
and takes values between $0$ and $1$ \citep{Loreau2008}. This variance ratio 
can be related to the classic variance ratio via
\begin{align}
\varphi &= \frac{\sum_{i,j} v_{ij}}{\sum_i v_{ii}} \\
&= \frac{\sum_{i,j} v_{ij}}{(\sum_i \sqrt{v_{ii}})^2}
\frac{(\sum_i \sqrt{v_{ii}})^2}{\sum_i v_{ii}} \\
&= \varphi^{(m)} \frac{(\sum_i \sqrt{v_{ii}})^2}{\sum_i v_{ii}}.
\end{align}
The quantity 
\begin{equation}
f=\frac{(\sum_i \sqrt{v_{ii}})^2}{\sum_i v_{ii}}
\end{equation}
takes values between $1$ and $S$ (see lemma \ref{lemJenson} below), 
and characterizes the degree to which the 
variances $v_{ii}$ are heterogeneous. For instance, when these variances are all
the same this term is $S$. When there is an index $i$ such that 
$v_{ii} \gg v_{jj}$ for all $j \neq i$, $f$ is close to $1$. 
The quantity $f$ does not really relate to synchrony,
as it does not depend on covariances or correlations between time series in
different locations. Thus $\varphi$ has information
about synchrony in it, but it has other information as well.

The previous relationship $\CV_{\com}^2=\varphi \CV_{\comip}^2$ becomes 
\begin{equation}
\CV_{\com}^2=\varphi^{(m)} f \CV_{\comip}^2.
\end{equation}
The quantity $\varphi^{(m)}$ is considered an index of synchrony
\citep{Loreau2008}, and this 
new equation makes it clear that not only are synchrony/compensatory dynamics
implicated in
to what extent population variability ($\CV_{\comip}^2$) 
translates into community variability ($\CV_{\com}^2$),
so is the quantity $f$, which represents the degree to which the variances
$v_{ii}$ are heterogeneous. But this makes sense because when these variances
are very different, the degree to which time series can reinforce each other (synchrony)
or cancel each other out (compensatory dynamics) is limited. Whereas 
when the variances $v_{ii}$ are similar,
time series can reinforce each other or cancel each other out substantially,
and $f$ is large and the effects of the overall variance ratio
$\varphi$ are accentuated. Following \cite{Loreau2008}, we define
$\CV_{\pop}^2 = f \CV_{\comip}^2 = \frac{\left( \sum_i \sqrt{v_{ii}} \right)^2}{\left( \sum_i \mu_i \right)^2}$, 
so that 
$\CV_{\com}^2 = \varphi^{(m)} \CV_{\pop}^2$. Then $\CV_{\pop}^2 = \frac{\left( \sum_i \sqrt{v_{ii}} \right)^2}{\left( \sum_i \mu_i  \right)^2}$.

The timescale-specific variance ratio can also be decomposed in a similar way,
\begin{equation}
\varphi_{\text{ts}}(\sigma)=\frac{\sum_{i,j}s_{ij}(\sigma)}{(\sum_i \sqrt{s_{ii}(\sigma)})^2}
\frac{(\sum_i \sqrt{s_{ii}(\sigma)})^2}{\sum_i s_{ii}(\sigma)}.
\end{equation}
The first factor on the right here is a timescale-specific version of 
$\varphi^{(m)}$, which we denote $\varphi_{\text{ts}}^{(m)}(\sigma)$. The new term
\begin{equation}
f(\sigma)=\frac{(\sum_i \sqrt{s_{ii}(\sigma)})^2}{\sum_i s_{ii}(\sigma)}
\end{equation}
is a timescale-specific version of $f$, with similar interpretation. We can define 
$\CV_{\pop}^2(\sigma)=f(\sigma) \CV_{\comip}^2 (\sigma)$, so that 
$\CV_{\com}^2(\sigma) = \varphi_{\text{ts}}^{(m)}(\sigma) \CV_{\pop}^2(\sigma)$.
Then $\CV_{\pop}^2(\sigma)=\frac{\left( \sum_i \sqrt{s_{ii}(\sigma)} \right)^2}{\left( \sum_i \mu_i  \right)^2}$. One would hope that the sum of this quantity across
timescales would equal $\CV_{\pop}^2$, but that is not the case because of the square root
and the square.
For this reason we have been unable to find a satisfactory extension of our 
timescale-specific approach to the Loreau-de Mazancourt variance ratio.

\begin{lemma}\label{lemJenson}
The quantity $f$ defined above satisfies $1 \leq f \leq S$.
\end{lemma}
\begin{proof}
\noindent It suffices to show 
$1 \leq \frac{(\sum_{i=1}^S a_i)^2}{\sum_{i=1}^S a_i^2} \leq S$
for positive quantities $a_i$. To see that 
$1 \leq \frac{(\sum_{i=1}^S a_i)^2}{\sum_{i=1}^S a_i^2}$, note that
\begin{align}
\left(\sum_i a_i \right)^2 &= \sum_{i,j} a_i a_j \\
 &= \sum_i a_i^2 + \sum_{i \neq j} a_i a_j \\
 &\geq \sum_i a_i^2.
\end{align}
To show the other inequality, it suffices to show
\begin{equation}
\left( \sum_i a_i  \right)^2 \leq S \sum_i a_i^2.
\end{equation}
Dividing by $S^2$, this is
\begin{equation}
\left( \frac{\sum_i a_i}{S}  \right)^2 \leq  \frac{\sum_i a_i^2}{S},
\end{equation}
i.e., the square of a mean is less than or equal to the mean of squares.
But that follows from Jensen's inequality.
\end{proof}

\bibliographystyle{plainnat}
\bibliography{SuppMatBib.bib}

\newpage
\clearpage

% latex table generated in R 3.5.1 by xtable 1.8-3 package
% Mon Sep 23 10:20:13 2019
\begin{table}[ht]
\centering
\scalebox{0.75}{
\begin{tabular}{lllllllll}
  \hline
Site & Abbr. & Years & Yr. rg. & Plots & Plot size & Richness & Measured & Description \\ 
  \hline
Jasper Ridge Biological Preserve & JRG &  28 & 1983-2010 &  18 & 0.80 & 25.70 & Percent cover & Serpentine grassland \\ 
  Kellogg Biological Station LTER & KBS &  11 & 1999-2009 &  30 & 1.00 & 34.50 & Biomass & Old field \\ 
  Hays, Kansas & HAY &  30 & 1943-1972 &  13 & 1.00 & 22.20 & Percent cover & Tallgrass prairie \\ 
  Jornada Basin LTER & JRN &  20 & 1989-2008 &  47 & 1.00 & 28.50 & Allometric biomass & Desert grassland \\ 
  Konza Prarie LTER & KNZ &  24 & 1983-2006 &  20 & 10.00 & 37.60 & Percent cover & Annually burned tallgrass prairie \\ 
  Sevilleta LTER & SEV &  13 & 1999-2011 &  22 & 1.00 & 13.40 & Biomass & Desert grassland \\ 
   \hline
\end{tabular}
}
\caption[Summary of datasets]{Summary of datasets. Plot size in square meters. Richness is the number of species that were ever seen in a plot, averaged across plots for a site. Biomass, when measured, was in g per square meter} 
\end{table}


% Figure
\begin{figure}
\includegraphics[width=0.9\textwidth]{Figs/fig_decomposed_1_classic_jrg_sorted_vrf.pdf}
\caption{Demonstration of timescale analyses of data from JRG, which consist of 18 plots (indicated by different colors in left panels, and by x-axis in right panels). Different colors in the right panels indicate the average (or weighted average) of the values in the corresponding left panels across short timescales ($<$ 4 years; blue) or long timescales ($\geq$ 4 years; red). On right panels, plots are sorted on the x-axis by the difference between short- and long-timescale averaged $\varphi _{ts} (\sigma)$.}
\end{figure}

\begin{figure}
\includegraphics[width=0.9\textwidth]{Figs/fig_decomposed_2_classic_kbs_sorted_vrf.pdf}
\caption{Demonstration of timescale analyses of data from KBS, which consist of 30 plots (indicated by different colors in left panels, and by x-axis in right panels). Different colors in the right panels indicate the average (or weighted average) of the values in the corresponding left panels across short timescales ($<$ 4 years; blue) or long timescales ($\geq$ 4 years; red). On right panels, plots are sorted on the x-axis by the difference between short- and long-timescale averaged $\varphi _{ts} (\sigma)$.}
\end{figure}

\begin{figure}
\includegraphics[width=0.9\textwidth]{Figs/fig_decomposed_3_classic_hay_sorted_vrf.pdf}
\caption{Demonstration of timescale analyses of data from HAY, which consist of 13 plots (indicated by different colors in left panels, and by x-axis in right panels). Different colors in the right panels indicate the average (or weighted average) of the values in the corresponding left panels across short timescales ($<$ 4 years; blue) or long timescales ($\geq$ 4 years; red). On right panels, plots are sorted on the x-axis by the difference between short- and long-timescale averaged $\varphi _{ts} (\sigma)$.}
\end{figure}

\begin{figure}
\includegraphics[width=0.9\textwidth]{Figs/fig_decomposed_4_classic_jrn_omit_sorted_vrf.pdf}
\caption{Demonstration of timescale analyses of data from JRN, which consist of 47 plots (indicated by different colors in left panels, and by x-axis in right panels). Different colors in the right panels indicate the average (or weighted average) of the values in the corresponding left panels across short timescales ($<$ 4 years; blue) or long timescales ($\geq$ 4 years; red). On right panels, plots are sorted on the x-axis by the difference between short- and long-timescale averaged $\varphi _{ts} (\sigma)$.}
\end{figure}

\begin{figure}
\includegraphics[width=0.9\textwidth]{Figs/fig_decomposed_5_classic_knz_sorted_vrf.pdf}
\caption{Demonstration of timescale analyses of data from KNZ, which consist of 20 plots (indicated by different colors in left panels, and by x-axis in right panels). Different colors in the right panels indicate the average (or weighted average) of the values in the corresponding left panels across short timescales ($<$ 4 years; blue) or long timescales ($\geq$ 4 years; red). On right panels, plots are sorted on the x-axis by the difference between short- and long-timescale averaged $\varphi _{ts} (\sigma)$.}
\end{figure}

\begin{figure}
\includegraphics[width=0.9\textwidth]{Figs/fig_decomposed_6_classic_sev_sorted_vrf.pdf}
\caption{Demonstration of timescale analyses of data from SEV, which consist of 22 plots (indicated by different colors in left panels, and by x-axis in right panels). Different colors in the right panels indicate the average (or weighted average) of the values in the corresponding left panels across short timescales ($<$ 4 years; blue) or long timescales ($\geq$ 4 years; red). On right panels, plots are sorted on the x-axis by the difference between short- and long-timescale averaged $\varphi _{ts} (\sigma)$.}
\end{figure}

\begin{figure}
\includegraphics[width=0.9\textwidth]{Figs/fig_timescale_quantile.pdf}
\caption{Distributions across plots within sites of timescale-specific variance ratios
$\varphi_{\text{ts}}(\sigma)$. The black line is the median of the distribution
across plots of $\varphi_{\text{ts}}(\sigma)$ for each timescale, $\sigma$. Dark-gray
shading covers $25^{\text{th}}$ and $75^{\text{th}}$ quantiles, and light-gray 
shading covers $2.5^{\text{th}}$ and $97.5^{\text{th}}$ quantiles. The horizontal 
dashed line is a variance ratio value of $1$.}
\end{figure}

\begin{figure}
\includegraphics[width=0.9\textwidth]{Figs/fig_new_vs_classic.pdf}
\caption{Relationship between the weighted average of $\varphi_{\text{ts}}(\sigma)$
across short (blue) and long (red) timescales and the classic, non-timescale-specific
variance ratio across plots for each site. The dashed line on each panel is the 1:1 
line.}
\end{figure}

\end{document}


